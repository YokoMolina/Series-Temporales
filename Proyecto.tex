% Options for packages loaded elsewhere
\PassOptionsToPackage{unicode}{hyperref}
\PassOptionsToPackage{hyphens}{url}
%
\documentclass[
]{article}
\usepackage{amsmath,amssymb}
\usepackage{iftex}
\ifPDFTeX
  \usepackage[T1]{fontenc}
  \usepackage[utf8]{inputenc}
  \usepackage{textcomp} % provide euro and other symbols
\else % if luatex or xetex
  \usepackage{unicode-math} % this also loads fontspec
  \defaultfontfeatures{Scale=MatchLowercase}
  \defaultfontfeatures[\rmfamily]{Ligatures=TeX,Scale=1}
\fi
\usepackage{lmodern}
\ifPDFTeX\else
  % xetex/luatex font selection
\fi
% Use upquote if available, for straight quotes in verbatim environments
\IfFileExists{upquote.sty}{\usepackage{upquote}}{}
\IfFileExists{microtype.sty}{% use microtype if available
  \usepackage[]{microtype}
  \UseMicrotypeSet[protrusion]{basicmath} % disable protrusion for tt fonts
}{}
\makeatletter
\@ifundefined{KOMAClassName}{% if non-KOMA class
  \IfFileExists{parskip.sty}{%
    \usepackage{parskip}
  }{% else
    \setlength{\parindent}{0pt}
    \setlength{\parskip}{6pt plus 2pt minus 1pt}}
}{% if KOMA class
  \KOMAoptions{parskip=half}}
\makeatother
\usepackage{xcolor}
\usepackage[margin=1in]{geometry}
\usepackage{color}
\usepackage{fancyvrb}
\newcommand{\VerbBar}{|}
\newcommand{\VERB}{\Verb[commandchars=\\\{\}]}
\DefineVerbatimEnvironment{Highlighting}{Verbatim}{commandchars=\\\{\}}
% Add ',fontsize=\small' for more characters per line
\usepackage{framed}
\definecolor{shadecolor}{RGB}{248,248,248}
\newenvironment{Shaded}{\begin{snugshade}}{\end{snugshade}}
\newcommand{\AlertTok}[1]{\textcolor[rgb]{0.94,0.16,0.16}{#1}}
\newcommand{\AnnotationTok}[1]{\textcolor[rgb]{0.56,0.35,0.01}{\textbf{\textit{#1}}}}
\newcommand{\AttributeTok}[1]{\textcolor[rgb]{0.13,0.29,0.53}{#1}}
\newcommand{\BaseNTok}[1]{\textcolor[rgb]{0.00,0.00,0.81}{#1}}
\newcommand{\BuiltInTok}[1]{#1}
\newcommand{\CharTok}[1]{\textcolor[rgb]{0.31,0.60,0.02}{#1}}
\newcommand{\CommentTok}[1]{\textcolor[rgb]{0.56,0.35,0.01}{\textit{#1}}}
\newcommand{\CommentVarTok}[1]{\textcolor[rgb]{0.56,0.35,0.01}{\textbf{\textit{#1}}}}
\newcommand{\ConstantTok}[1]{\textcolor[rgb]{0.56,0.35,0.01}{#1}}
\newcommand{\ControlFlowTok}[1]{\textcolor[rgb]{0.13,0.29,0.53}{\textbf{#1}}}
\newcommand{\DataTypeTok}[1]{\textcolor[rgb]{0.13,0.29,0.53}{#1}}
\newcommand{\DecValTok}[1]{\textcolor[rgb]{0.00,0.00,0.81}{#1}}
\newcommand{\DocumentationTok}[1]{\textcolor[rgb]{0.56,0.35,0.01}{\textbf{\textit{#1}}}}
\newcommand{\ErrorTok}[1]{\textcolor[rgb]{0.64,0.00,0.00}{\textbf{#1}}}
\newcommand{\ExtensionTok}[1]{#1}
\newcommand{\FloatTok}[1]{\textcolor[rgb]{0.00,0.00,0.81}{#1}}
\newcommand{\FunctionTok}[1]{\textcolor[rgb]{0.13,0.29,0.53}{\textbf{#1}}}
\newcommand{\ImportTok}[1]{#1}
\newcommand{\InformationTok}[1]{\textcolor[rgb]{0.56,0.35,0.01}{\textbf{\textit{#1}}}}
\newcommand{\KeywordTok}[1]{\textcolor[rgb]{0.13,0.29,0.53}{\textbf{#1}}}
\newcommand{\NormalTok}[1]{#1}
\newcommand{\OperatorTok}[1]{\textcolor[rgb]{0.81,0.36,0.00}{\textbf{#1}}}
\newcommand{\OtherTok}[1]{\textcolor[rgb]{0.56,0.35,0.01}{#1}}
\newcommand{\PreprocessorTok}[1]{\textcolor[rgb]{0.56,0.35,0.01}{\textit{#1}}}
\newcommand{\RegionMarkerTok}[1]{#1}
\newcommand{\SpecialCharTok}[1]{\textcolor[rgb]{0.81,0.36,0.00}{\textbf{#1}}}
\newcommand{\SpecialStringTok}[1]{\textcolor[rgb]{0.31,0.60,0.02}{#1}}
\newcommand{\StringTok}[1]{\textcolor[rgb]{0.31,0.60,0.02}{#1}}
\newcommand{\VariableTok}[1]{\textcolor[rgb]{0.00,0.00,0.00}{#1}}
\newcommand{\VerbatimStringTok}[1]{\textcolor[rgb]{0.31,0.60,0.02}{#1}}
\newcommand{\WarningTok}[1]{\textcolor[rgb]{0.56,0.35,0.01}{\textbf{\textit{#1}}}}
\usepackage{graphicx}
\makeatletter
\def\maxwidth{\ifdim\Gin@nat@width>\linewidth\linewidth\else\Gin@nat@width\fi}
\def\maxheight{\ifdim\Gin@nat@height>\textheight\textheight\else\Gin@nat@height\fi}
\makeatother
% Scale images if necessary, so that they will not overflow the page
% margins by default, and it is still possible to overwrite the defaults
% using explicit options in \includegraphics[width, height, ...]{}
\setkeys{Gin}{width=\maxwidth,height=\maxheight,keepaspectratio}
% Set default figure placement to htbp
\makeatletter
\def\fps@figure{htbp}
\makeatother
\setlength{\emergencystretch}{3em} % prevent overfull lines
\providecommand{\tightlist}{%
  \setlength{\itemsep}{0pt}\setlength{\parskip}{0pt}}
\setcounter{secnumdepth}{-\maxdimen} % remove section numbering
\ifLuaTeX
  \usepackage{selnolig}  % disable illegal ligatures
\fi
\IfFileExists{bookmark.sty}{\usepackage{bookmark}}{\usepackage{hyperref}}
\IfFileExists{xurl.sty}{\usepackage{xurl}}{} % add URL line breaks if available
\urlstyle{same}
\hypersetup{
  pdftitle={Proyecto},
  pdfauthor={Fabián Encarnación, Geoconda Molina},
  hidelinks,
  pdfcreator={LaTeX via pandoc}}

\title{Proyecto}
\author{Fabián Encarnación, Geoconda Molina}
\date{2023-12-12}

\begin{document}
\maketitle

\begin{Shaded}
\begin{Highlighting}[]
\FunctionTok{library}\NormalTok{(readxl)}
\FunctionTok{library}\NormalTok{(tidyverse)}
\end{Highlighting}
\end{Shaded}

\begin{verbatim}
## -- Attaching core tidyverse packages ------------------------ tidyverse 2.0.0 --
## v dplyr     1.1.4     v readr     2.1.4
## v forcats   1.0.0     v stringr   1.5.0
## v ggplot2   3.4.2     v tibble    3.2.1
## v lubridate 1.9.3     v tidyr     1.3.0
## v purrr     1.0.1     
## -- Conflicts ------------------------------------------ tidyverse_conflicts() --
## x dplyr::filter() masks stats::filter()
## x dplyr::lag()    masks stats::lag()
## i Use the conflicted package (<http://conflicted.r-lib.org/>) to force all conflicts to become errors
\end{verbatim}

\begin{Shaded}
\begin{Highlighting}[]
\FunctionTok{library}\NormalTok{(tsibble)}
\end{Highlighting}
\end{Shaded}

\begin{verbatim}
## 
## Attaching package: 'tsibble'
## 
## The following object is masked from 'package:lubridate':
## 
##     interval
## 
## The following objects are masked from 'package:base':
## 
##     intersect, setdiff, union
\end{verbatim}

\begin{Shaded}
\begin{Highlighting}[]
\FunctionTok{library}\NormalTok{(fabletools)}
\FunctionTok{library}\NormalTok{(forecast)}
\end{Highlighting}
\end{Shaded}

\begin{verbatim}
## Registered S3 method overwritten by 'quantmod':
##   method            from
##   as.zoo.data.frame zoo
\end{verbatim}

\begin{Shaded}
\begin{Highlighting}[]
\FunctionTok{library}\NormalTok{(fpp3)}
\end{Highlighting}
\end{Shaded}

\begin{verbatim}
## -- Attaching packages ---------------------------------------------- fpp3 0.5 --
## v tsibbledata 0.4.1     v fable       0.3.3
## v feasts      0.3.1     
## -- Conflicts ------------------------------------------------- fpp3_conflicts --
## x lubridate::date()    masks base::date()
## x dplyr::filter()      masks stats::filter()
## x tsibble::intersect() masks base::intersect()
## x tsibble::interval()  masks lubridate::interval()
## x dplyr::lag()         masks stats::lag()
## x tsibble::setdiff()   masks base::setdiff()
## x tsibble::union()     masks base::union()
\end{verbatim}

\begin{Shaded}
\begin{Highlighting}[]
\NormalTok{data }\OtherTok{\textless{}{-}} \FunctionTok{read.csv}\NormalTok{(}\StringTok{\textquotesingle{}train.csv\textquotesingle{}}\NormalTok{,}\AttributeTok{sep =} \StringTok{";"}\NormalTok{) }
\CommentTok{\#ELEGIMOS STORE 13, Dept 1, YA QUE ES LA DE MAYOR SIZE }
\NormalTok{data13 }\OtherTok{\textless{}{-}}\NormalTok{ data }\SpecialCharTok{\%\textgreater{}\%} \FunctionTok{filter}\NormalTok{(data}\SpecialCharTok{$}\NormalTok{Store}\SpecialCharTok{==}\DecValTok{13} \SpecialCharTok{\&}\NormalTok{ data}\SpecialCharTok{$}\NormalTok{Dept}\SpecialCharTok{==}\DecValTok{4}\NormalTok{)}
\NormalTok{dataf }\OtherTok{\textless{}{-}} \FunctionTok{read.csv}\NormalTok{(}\StringTok{\textquotesingle{}features.csv\textquotesingle{}}\NormalTok{,}\AttributeTok{sep =} \StringTok{";"}\NormalTok{)}
\NormalTok{dataf13 }\OtherTok{\textless{}{-}}\NormalTok{ dataf }\SpecialCharTok{\%\textgreater{}\%} \FunctionTok{filter}\NormalTok{(dataf}\SpecialCharTok{$}\NormalTok{Store}\SpecialCharTok{==}\DecValTok{13}\NormalTok{)}

\CommentTok{\#View(data)}

\CommentTok{\#Variable dependiente}
\NormalTok{yt }\OtherTok{\textless{}{-}}\NormalTok{ data13 }\SpecialCharTok{\%\textgreater{}\%} \FunctionTok{mutate}\NormalTok{(}\AttributeTok{Date=}\FunctionTok{as.Date}\NormalTok{(Date,}\StringTok{"\%d/\%m/\%y"}\NormalTok{))}\SpecialCharTok{\%\textgreater{}\%} \FunctionTok{select}\NormalTok{(Date , Weekly\_Sales)}\SpecialCharTok{\%\textgreater{}\%} \FunctionTok{as\_tsibble}\NormalTok{(}\AttributeTok{index=}\NormalTok{Date)}
\FunctionTok{autoplot}\NormalTok{(yt,}\AttributeTok{.vars =}\NormalTok{ Weekly\_Sales)}
\end{Highlighting}
\end{Shaded}

\includegraphics{Proyecto_files/figure-latex/unnamed-chunk-2-1.pdf}

\begin{Shaded}
\begin{Highlighting}[]
\CommentTok{\#Variable indep 1}
\NormalTok{x1t }\OtherTok{\textless{}{-}}\NormalTok{ dataf13 }\SpecialCharTok{\%\textgreater{}\%} \FunctionTok{mutate}\NormalTok{(}\AttributeTok{Date=}\FunctionTok{as.Date}\NormalTok{(Date,}\StringTok{"\%d/\%m/\%Y"}\NormalTok{)) }\SpecialCharTok{\%\textgreater{}\%} \FunctionTok{select}\NormalTok{(Date,Temperature) }\SpecialCharTok{\%\textgreater{}\%} \FunctionTok{as\_tsibble}\NormalTok{(}\AttributeTok{index =}\NormalTok{ Date)}
\FunctionTok{autoplot}\NormalTok{(x1t,}\AttributeTok{.vars =}\NormalTok{ Temperature)}
\end{Highlighting}
\end{Shaded}

\includegraphics{Proyecto_files/figure-latex/unnamed-chunk-2-2.pdf}

\begin{Shaded}
\begin{Highlighting}[]
\CommentTok{\# Variable indep 2}

\NormalTok{x2t }\OtherTok{\textless{}{-}}\NormalTok{ dataf13 }\SpecialCharTok{\%\textgreater{}\%} \FunctionTok{mutate}\NormalTok{(}\AttributeTok{Date=}\FunctionTok{as.Date}\NormalTok{(Date,}\StringTok{"\%d/\%m/\%Y"}\NormalTok{)) }\SpecialCharTok{\%\textgreater{}\%}\FunctionTok{select}\NormalTok{(Date,Fuel\_Price)}\SpecialCharTok{\%\textgreater{}\%} \FunctionTok{as\_tsibble}\NormalTok{(}\AttributeTok{index =}\NormalTok{ Date)}

\FunctionTok{autoplot}\NormalTok{(x2t,}\AttributeTok{.vars=}\NormalTok{Fuel\_Price)}
\end{Highlighting}
\end{Shaded}

\includegraphics{Proyecto_files/figure-latex/unnamed-chunk-2-3.pdf}

\begin{Shaded}
\begin{Highlighting}[]
\CommentTok{\# Variable indep 2}

\NormalTok{x3t }\OtherTok{\textless{}{-}}\NormalTok{ dataf13 }\SpecialCharTok{\%\textgreater{}\%} \FunctionTok{mutate}\NormalTok{(}\AttributeTok{Date=}\FunctionTok{as.Date}\NormalTok{(Date,}\StringTok{"\%d/\%m/\%Y"}\NormalTok{))}\SpecialCharTok{\%\textgreater{}\%} \FunctionTok{select}\NormalTok{(Date,CPI)}\SpecialCharTok{\%\textgreater{}\%} \FunctionTok{as\_tsibble}\NormalTok{(}\AttributeTok{index =}\NormalTok{ Date)}

\FunctionTok{plot}\NormalTok{(x3t,}\AttributeTok{type=} \StringTok{"l"}\NormalTok{)}
\end{Highlighting}
\end{Shaded}

\begin{verbatim}
## Warning in xy.coords(x, y, xlabel, ylabel, log): NAs introducidos por coerción
\end{verbatim}

\includegraphics{Proyecto_files/figure-latex/unnamed-chunk-2-4.pdf}

\begin{Shaded}
\begin{Highlighting}[]
\NormalTok{x4t }\OtherTok{\textless{}{-}}\NormalTok{ dataf13 }\SpecialCharTok{\%\textgreater{}\%} \FunctionTok{mutate}\NormalTok{(}\AttributeTok{Date=}\FunctionTok{as.Date}\NormalTok{(Date,}\StringTok{"\%d/\%m/\%Y"}\NormalTok{)) }\SpecialCharTok{\%\textgreater{}\%} \FunctionTok{select}\NormalTok{(Date,Unemployment)}\SpecialCharTok{\%\textgreater{}\%} \FunctionTok{as\_tsibble}\NormalTok{(}\AttributeTok{index =}\NormalTok{ Date)}
\FunctionTok{plot}\NormalTok{(x4t,}\AttributeTok{type=}\StringTok{\textquotesingle{}l\textquotesingle{}}\NormalTok{)}
\end{Highlighting}
\end{Shaded}

\includegraphics{Proyecto_files/figure-latex/unnamed-chunk-2-5.pdf}

\begin{Shaded}
\begin{Highlighting}[]
\NormalTok{x5t }\OtherTok{\textless{}{-}}\NormalTok{ dataf13 }\SpecialCharTok{\%\textgreater{}\%} \FunctionTok{mutate}\NormalTok{(}\AttributeTok{Date=}\FunctionTok{as.Date}\NormalTok{(Date,}\StringTok{"\%d/\%m/\%Y"}\NormalTok{)) }\SpecialCharTok{\%\textgreater{}\%} \FunctionTok{select}\NormalTok{(Date,IsHoliday)}\SpecialCharTok{\%\textgreater{}\%} \FunctionTok{as\_tsibble}\NormalTok{(}\AttributeTok{index =}\NormalTok{ Date)}
\FunctionTok{plot}\NormalTok{(x5t,}\AttributeTok{type=}\StringTok{\textquotesingle{}l\textquotesingle{}}\NormalTok{)}
\end{Highlighting}
\end{Shaded}

\includegraphics{Proyecto_files/figure-latex/unnamed-chunk-2-6.pdf}

\hypertarget{variable-weekly_sales}{%
\subsection{Variable Weekly\_Sales}\label{variable-weekly_sales}}

\begin{Shaded}
\begin{Highlighting}[]
\CommentTok{\#Variable 1}
\NormalTok{yt }\OtherTok{\textless{}{-}}\NormalTok{ data13 }\SpecialCharTok{\%\textgreater{}\%} \FunctionTok{mutate}\NormalTok{(}\AttributeTok{Date=}\FunctionTok{as.Date}\NormalTok{(Date,}\StringTok{"\%d/\%m/\%Y"}\NormalTok{))}\SpecialCharTok{\%\textgreater{}\%} \FunctionTok{select}\NormalTok{(Date , Weekly\_Sales)}\SpecialCharTok{\%\textgreater{}\%} \FunctionTok{as\_tsibble}\NormalTok{(}\AttributeTok{index=}\NormalTok{Date)}

\FunctionTok{autoplot}\NormalTok{(yt,}\AttributeTok{.vars =}\NormalTok{ Weekly\_Sales)}
\end{Highlighting}
\end{Shaded}

\includegraphics{Proyecto_files/figure-latex/unnamed-chunk-3-1.pdf}

\begin{Shaded}
\begin{Highlighting}[]
\CommentTok{\# 1) Ajustemos un modelo AR(p) a la serie de tiempo "ventas semanales"}

\CommentTok{\#View(data13)}


\NormalTok{modelo\_1 }\OtherTok{\textless{}{-}}\NormalTok{ yt }\SpecialCharTok{\%\textgreater{}\%}
  \FunctionTok{model}\NormalTok{(}\FunctionTok{ARIMA}\NormalTok{(Weekly\_Sales }\SpecialCharTok{\textasciitilde{}} \FunctionTok{pdq}\NormalTok{(}\DecValTok{2}\NormalTok{,}\DecValTok{0}\NormalTok{,}\DecValTok{0}\NormalTok{))) }


\FunctionTok{report}\NormalTok{(modelo\_1)}
\end{Highlighting}
\end{Shaded}

\begin{verbatim}
## Series: Weekly_Sales 
## Model: ARIMA(2,0,0) w/ mean 
## 
## Coefficients:
##          ar1     ar2    constant
##       0.3081  0.3943  12664.7571
## s.e.  0.0761  0.0763    225.0971
## 
## sigma^2 estimated as 7781721:  log likelihood=-1336.22
## AIC=2680.44   AICc=2680.73   BIC=2692.3
\end{verbatim}

\begin{Shaded}
\begin{Highlighting}[]
\FunctionTok{gg\_tsresiduals}\NormalTok{(modelo\_1)}
\end{Highlighting}
\end{Shaded}

\includegraphics{Proyecto_files/figure-latex/unnamed-chunk-3-2.pdf}

\begin{Shaded}
\begin{Highlighting}[]
\CommentTok{\# 2) Ajustemos un modelo ARMA(p,q) a la serie de tiempo "ventas semanales"}

\NormalTok{modelo\_2 }\OtherTok{\textless{}{-}}\NormalTok{ yt }\SpecialCharTok{\%\textgreater{}\%}
  \FunctionTok{model}\NormalTok{(}\FunctionTok{ARIMA}\NormalTok{(Weekly\_Sales }\SpecialCharTok{\textasciitilde{}} \FunctionTok{pdq}\NormalTok{(}\DecValTok{3}\NormalTok{,}\DecValTok{0}\NormalTok{,}\DecValTok{2}\NormalTok{))) }


\FunctionTok{report}\NormalTok{(modelo\_2)}
\end{Highlighting}
\end{Shaded}

\begin{verbatim}
## Series: Weekly_Sales 
## Model: ARIMA(3,0,2) w/ mean 
## 
## Coefficients:
##          ar1     ar2      ar3     ma1      ma2   constant
##       0.2427  0.7374  -0.0915  0.0020  -0.4105  4729.1293
## s.e.  0.2342  0.1164   0.1403  0.2184   0.1174   122.3142
## 
## sigma^2 estimated as 7396002:  log likelihood=-1331.15
## AIC=2676.3   AICc=2677.13   BIC=2697.04
\end{verbatim}

\begin{Shaded}
\begin{Highlighting}[]
\FunctionTok{gg\_tsresiduals}\NormalTok{(modelo\_2)}
\end{Highlighting}
\end{Shaded}

\includegraphics{Proyecto_files/figure-latex/unnamed-chunk-3-3.pdf}

\begin{Shaded}
\begin{Highlighting}[]
\CommentTok{\# 3) Ajustemos un modelo ARIMA(p,d,q) a la serie de tiempo "ventas semanales"}


\NormalTok{modelo\_3 }\OtherTok{\textless{}{-}}\NormalTok{ yt }\SpecialCharTok{\%\textgreater{}\%}
  \FunctionTok{model}\NormalTok{(}\FunctionTok{ARIMA}\NormalTok{(Weekly\_Sales }\SpecialCharTok{\textasciitilde{}} \FunctionTok{pdq}\NormalTok{(}\DecValTok{3}\NormalTok{,}\DecValTok{2}\NormalTok{,}\DecValTok{2}\NormalTok{))) }


\FunctionTok{report}\NormalTok{(modelo\_3)}
\end{Highlighting}
\end{Shaded}

\begin{verbatim}
## Series: Weekly_Sales 
## Model: ARIMA(3,2,2) 
## 
## Coefficients:
##           ar1      ar2      ar3      ma1      ma2
##       -0.8349  -0.5058  -0.4090  -0.8496  -0.1504
## s.e.   0.1616   0.1282   0.0786   0.1710   0.1699
## 
## sigma^2 estimated as 7207221:  log likelihood=-1314.56
## AIC=2641.12   AICc=2641.75   BIC=2658.81
\end{verbatim}

\begin{Shaded}
\begin{Highlighting}[]
\FunctionTok{gg\_tsresiduals}\NormalTok{(modelo\_3)}
\end{Highlighting}
\end{Shaded}

\includegraphics{Proyecto_files/figure-latex/unnamed-chunk-3-4.pdf}

\end{document}
